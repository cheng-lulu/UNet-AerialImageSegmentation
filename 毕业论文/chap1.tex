\begin{spacing}{2}
    \section{绪论}
\end{spacing}
\subsection{研究背景与意义}
遥感探测技术是一种使用卫星搭载的传感器,对地球电磁波谱段成像的技术。遥感探测技术可以在极短的时间内进行大范围的观测,是一种高效地获取地球信息的手段。遥感探测技术的高低可以衡量一个国家的科技水平,因此我国对遥感探测技术的发展十分重视。对于个人来说,我们需要借助地图导航及定位。对于国家来说,需要使用遥感图像来进行国土资源统计或者土地规划。传统遥感图像依赖人工标注出道路、桥梁、建筑、土地,工作量大,耗时长。如果能借由计算机自动划分道路、建筑、耕地,则可以简化诸如资源统计、土地规划等工作流程,增加工作效率。因此遥感图像自动分割技术有着广泛的应用空间。人工智能近年来来计算机视觉领域取得了巨大的进展。如果能将现代先进的深度学习技术与遥感图像相结合,开发出一种自动识别地物的算法,可以更高效地挖掘、利用遥感图像信息。

遥感影像的分辨率随着技术的发展越来越高。在分辨率不高的时代,遥感影像分类大多数是基于像元的。随着分辨率的提高,基于像元方法的劣势逐渐显现,因此出现了面向对象的分类方法。但是遥感影像中,地物的尺度不好确定,所以又出现了从语义级别分类的基于场景的分割方法。这三种分类方法在深度学习方法未出现之前的分类器构建流程为人工设计分类特征,然后使用机器学习方法进行分类。深度学习方法较传统方法的优势在于,能够自动提取特征且特征的表征能力强于人工设计的特征。

一开始人们尝试使用深度信念网络对遥感影像进行分类,之后使用卷积神经网络。但是这两种方法计算量大,分割效果不佳。在2015年出现了全卷积神经网络对图像进行像素级语义分割。全卷积神经网络计算量小且能接受任意尺寸的输入图像。U-Net是一种由全卷积神经网络启发的对称结构网络,在医疗影像分割领域取得了很好的效果。

此次研究尝试使用U-Net网络在对多光谱遥感影像数据集上进行训练,尝试使用卷积神经网络自动分割出建筑,希望能够得到一种自动分割遥感影像的简便方法。
\subsection{国内外研究现状}
自计算机用于图像处理以来,图像分割一直是一个受到广泛关注的问题。语义分割在图像分割的基础上对分割的内容提出了要包含语义信息的要求。遥感图像较以往的自然场景下的图像有相近处也有些许不同。
\subsubsection{语义分割研究现状}


\paragraph{基于特征的语义分割方法}
图像分割问题从计算机被用于处理的那天开始就是图像处理领域的一个核心问题。基于特征的图像分割技术主要被分为以下五个类别。

第一种方法是基于阈值的分割方法。图像分割技术在计算机发展早期受到计算机性能的限制,只能依据图像的底层特征对图像进行分割。这种方法的核心思想是利用图像的灰度特征。我们将设置一个或多个超参数:灰度阈值,通过将图像中的像素与我们设定的灰度阈值进行比较,逐一将像素分割到其合适的类别。这种方法可以分为自动设定灰度阈值的方法,如大津二值化算法\cite{otsu1979threshold}和手动设置灰度阈值的方法两种。第二种方法是基于边缘的分割方法。在图像两个不同区域的边界上,纹理、灰度和颜色等图像特征会产生突变。根据这一思想,基于边缘的分割方法主要关注边缘处的特征突变。根据这一突变对图像进行分割。第三种方法是基于区域的分割方法。该类方法中最著名的方法是归一化割\cite{shi2000normalized}和分水岭算法\cite{vincent1991watersheds}。这种分割方法主要依赖图像的相似性将图像划分为不同区域块。第四种基于图论的方法把图像分割问题看做图的最小割问题。这种方法的目标是最大化子图内部的相似度同时最小化子图之间的相似度。第五种方法是基于能量泛函的分割方法\cite{lankton2008localizing}。这种方法是一类以活动轮廓模型为基础,在此基础上演变、发展出来的一系列算法。这一系列算法的主要方法是利用连续曲线来表示目标的边缘并确定能量泛函,使其自变量变量包括边缘曲线。这样图像分割的过程就转变为了最小化这个能量泛函的过程。

\subparagraph{传统机器学习语义分割方法}
除此之外,机器学习不仅仅是在近几年才应用于图像分割任务上的。机器学习中的聚类算法如K-means\cite{ray1999determination}、谱聚类\cite{zelnik2005self}等也被尝试应用于图像分割任务。通常的思路是利用诸如颜色、亮度、纹理等特征对像素点进行聚类。

语义分割则对图像分割做了更高的要求,即分割的内容具有语义可解释性。虽然像聚类这样的无监督机器学习方法可以用于图像分割,但分割的结果不一定有语义。较简单的图像分割任务相比,语义分割能是我们对图像有更加细致的了解。传统的图像分割算法中主要是基于底层的特征,而基于语义的分割算法可以基于图像中高层内容信息对图像进行分割。因此语义分割算法在对结构复杂,内部差异性大的物体进行分割的时候表现较传统图像分割技术好。遥感图像中,既有道路、森林等有相对固定的纹理的物体,也有建筑、港口等较为复杂的物体。对于简单的物体的分割,传统的基于图论的方法和基于像素聚类的方法可能会对我们有所启示;而那些复杂的物体基于语义图像分割技术可能会有更好的效果。

深度学习方法在2012年AlexNet\cite{krizhevsky2012imagenet}发表之后取得了巨大的进展,被用于多个计算机视觉领域,其中就包括图像分割领域。由于卷积神经网络提取高层语义信息的能力,利用根据卷积神经网络进行图像分割也可以包含语义信息,达到图像语义分割的功能。

最初Ciresan等人\cite{ciresan2012deep}使用滑动窗口法对图像进行语义分割。但这种方法效率较低。2015年Long等人\cite{long2015fully}提出全卷积神经网络的概念。这种神经网络中只包含卷积层和池化层,可以对图像进行像素级语义分割,且开销小,效果好,称为现代图像分割的基础。U-Net网络就是基于这种方法得到的针对医疗图像分割的算法。

\subsubsection{将深度学习应用于遥感图像分割现状}
在早期也有将深度学习算法应用于遥感图像分割的方法。如Mitra等人使用支持向量机方法对多光谱遥感图像进行分割,在此之后与非监督方法的效果进行了比较\cite{mitra2004segmentation}。Su等人提出使用粒子群算法对地物进行分类,并且对粒子群方法应用于大规模遥感图像上的复杂度进行了评估\cite{su2014optimized}。在国内有徐涵秋等人\cite{徐涵秋2005利用改进的归一化差异水体指数}使用水体指数分割水域。肖鹏峰等人\cite{肖鹏峰2007基于相位一致的高分辨率遥感图像分割方法}使用小波提取图像的多尺度梯度以改善分水岭算法图像分割的效果。

在此之后,也有人使用深度学习的方法对分割遥感图像做出了尝试。如Mnih等人使用滑动窗口法分割道路和房屋\cite{mnih2013machine}。Marmanis等人\cite{marmanis2018classification}除了使用深度卷积神经网络对遥感图像进行语义分割之外还使用了SegNet对遥感图像进行边缘检测来增强分割效果。党宇等人在\cite{党宇2017基于深度学习}中使用deeplab自动提取耕地区域以监测耕地变化。左童春等人在\cite{左童春2017基于高分辨率}中提出了基于FCN网络的HF-FCN提取遥感图像中的建筑物。张永宏等人在\cite{张永宏2018基于全卷积神经网络的多源高分辨率遥感道路提取}使用FCN网络提取日喀则地区道路的精度达到了99.2\%。
\subsection{本文的主要工作}
本文首先提出了一种基于遥感图像类别比率的交叉熵损失函数——类别平衡交叉熵。并与应用于医疗图像分割的U-Net相结合,将其应用于遥感图像语义分割。在Inria Aerial Image Labeling Dataset训练数据集上分别使用交叉熵损失函数和类别平衡交叉熵损失函数进行训练,得到两个训练好的卷积神经网络。再利用这两个网络在Inria Aerial Image Labeling Dataset测试数据集上生成预测图像进行比对。
\subsection{论文章节安排}
本文分为五章,章节内容如下:

第一章介绍了研究背景和国内外使用深度学习对遥感图像进行语义分割的现状,并介绍了本文的组织结构。第二章介绍了深度学习应用于语义分割的原理。第三章介绍了实验的流程和实验的各项参数设置。第四章分析了实验的结果。第五章对全文做出了总结和对未来做出了展望。